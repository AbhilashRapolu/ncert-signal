\iffalse
\let\negmedspace\undefined
\let\negthickspace\undefined
\documentclass[journal,12pt,twocolumn]{IEEEtran}
\usepackage{cite}
\usepackage{amsmath,amssymb,amsfonts,amsthm}
\usepackage{algorithmic}
\usepackage{graphicx}
\usepackage{textcomp}
\usepackage{xcolor}
\usepackage{txfonts}
\usepackage{listings}
\usepackage{enumitem}
\usepackage{mathtools}
\usepackage{gensymb}
\usepackage[breaklinks=true]{hyperref}
\usepackage{tkz-euclide} % loads  TikZ and tkz-base
\usepackage{listings}
\usepackage{gvv}


        

\newtheorem{theorem}{Theorem}[section]
\newtheorem{problem}{Problem}
\newtheorem{proposition}{Proposition}[section]
\newtheorem{lemma}{Lemma}[section]
\newtheorem{corollary}[theorem]{Corollary}
\newtheorem{example}{Example}[section]
\newtheorem{definition}[problem]{Definition}
\newcommand{\BEQA}{\begin{eqnarray}}
\newcommand{\EEQA}{\end{eqnarray}}
\newcommand{\define}{\stackrel{\triangle}{=}}
\theoremstyle{remark}
\newtheorem{rem}{Remark}

%\bibliographystyle{ieeetr}
\begin{document}
%

\bibliographystyle{IEEEtran}


\vspace{3cm}

\title{
%	\logo{
Discrete

\large{EE1205 : Signals and Systems}

Indian Institute of Technology Hyderabad
%	}
}
\author{Chirag Garg

(EE23BTECH11206)
}	





\maketitle

\newpage



\bigskip

\renewcommand{\thefigure}{\arabic{figure}}
\renewcommand{\thetable}{\arabic{table}}


\section{Question 11.9.5 (18)}
\vspace{0.5cm}
\begin{flushleft}
If $a$ and $b$ are the roots of $x^{2} -3x + p = 0$ and $c$ , $d$ are roots of $x^{2} - 12x + q = 0$ where $a,b,c,d$ form a G.P. Prove that $(q+p) : (q-p)$ = 17:15 .
\end{flushleft}  


\vspace{0.8cm}


\section{Solution} 
\fi
\begin{table}[htbp]
\centering
\resizebox{\columnwidth}{!}{
\begin{tabular}{|c|c|c|}
    \hline
     \textbf{Parameter} & \textbf{Value} &
     \textbf{Description}\\
    \hline 
     $x_1(n)$ &  - & G.P. Sequence\\
    \hline 
    $x_1(0)$ & $a$ & First term of G.P. \\
    \hline
     $x_1(1)$ & $b$ & Second term of G.P. \\
    \hline
     $x_1(2)$ & $c$ & Third term of G.P. \\
    \hline
   $x_1(3)$ & $d$ & Fourth term of G.P. \\
    \hline
    $r$ & $\dfrac{b}{a}$ & Common ratio \\
   
    \hline
								      
\end{tabular}}
\caption{ Given Parameters}

\end{table}

Given $x_1(0)$ and $x_1(1)$ are the roots of $x^{2} – 3x + p = 0$
So, we have :
\begin{align}
 a+b = 3 \label{eq:cg1}\\ 
 ab = p \label{eq:cg2}
\end{align}

Also, $x_1(2)$ and $x_1(3)$ are the roots of $x^{2} – 12x + q = 0$ , so,
\begin{align}
c+d&= 12 \label{eq:cg3}\\
cd &= q \label{eq:cg4}
 \end{align}  

From \ref{eq:cg1} and \ref{eq:cg3} , we get , 
\begin{align}
a(1+r)=3 \label{eq:cg5}
\end{align}
 
 And ,
 \begin{align}
ar^{2}(1+r) = 12 \label{eq:cg6}
\end{align}
On dividing eq. \ref{eq:cg5} and eq. \ref{eq:cg6}, we get
\begin{align}
\dfrac{ar^{2}(1+r)}{a(1+r)} &= \dfrac{12}{3} \\
r^{2} &= 4 \\
r &= \pm2
\end{align}
When r = 2, $a$ = 1 \\
When r = -2, $a$= -3 \\
Case 1 :
When $r = 2$ and $a = 1$ \\
\begin{align}
p &= ab\\
p&=2 \\
q&= cd \\
q&=32 \\
\dfrac{q+p}{q-p} &= \dfrac{32 + 2}{32 - 2} \\
&=\dfrac{17}{15}
\end{align}
Case 2 :
When $r = -2$ and $a = -3$ \\
\begin{align}
p &= ab\\
p&=-18 \\
q&= cd \\
q&=288 \\
\dfrac{q+p}{q-p} &= \dfrac{288-18}{288+18} \\
&=\dfrac{135}{153}
\end{align}
Hence , case 1 satisfies the condition .
\begin{center}
 $x_1(n) \longleftrightarrow X(z)$
\end{center}
 \begin{align}
ar^{n}u(n)  \longleftrightarrow  \dfrac{a}{1 - rz^{-1}} \; ; \; |z| > |r| 
 \end{align}
  \begin{align}
\therefore \; X(z) &= \dfrac{1}{1 - 2z^{-1}} \; ; \;( |z| > 2 )
 \end{align}
