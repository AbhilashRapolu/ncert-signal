\iffalse
\let\negmedspace\undefined
\let\negthickspace\undefined
\documentclass[journal,12pt,twocolumn]{IEEEtran}
\usepackage{circuitikz}
\usepackage{cite}
\usepackage{amsmath,amssymb,amsfonts,amsthm}
\usepackage{algorithmic}
\usepackage{graphicx}
\usepackage{textcomp}
\usepackage{xcolor}
\usepackage{txfonts}
\usepackage{listings}
\usepackage{enumitem}
\usepackage{mathtools}
\usepackage{gensymb}
\usepackage{comment}
\usepackage[breaklinks=true]{hyperref}
\usepackage{tkz-euclide} 
\usepackage{listings}
\usepackage{gvv}                                        
\def\inputGnumericTable{}                                 
\usepackage[latin1]{inputenc}                                
\usepackage{color}                                            
\usepackage{array}                                            
\usepackage{longtable}                                       
\usepackage{calc}                                             
\usepackage{multirow}                                         
\usepackage{hhline}                                           
\usepackage{ifthen}                                           
\usepackage{lscape}


\newtheorem{theorem}{Theorem}[section]
\newtheorem{problem}{Problem}
\newtheorem{proposition}{Proposition}[section]
\newtheorem{lemma}{Lemma}[section]
\newtheorem{corollary}[theorem]{Corollary}
\newtheorem{example}{Example}[section]
\newtheorem{definition}[problem]{Definition}
\newcommand{\BEQA}{\begin{eqnarray}}
\newcommand{\EEQA}{\end{eqnarray}}
\newcommand{\define}{\stackrel{\triangle}{=}}
\theoremstyle{remark}
\newtheorem{rem}{Remark}
\begin{document}
\parindent 0px
\bibliographystyle{IEEEtran}

\title{Assignment\\[1ex]12.7 - 8}
\author{EE23BTECH11220 - R.V.S.S Varun$^{}$% <-this % stops a space
}
\maketitle
\newpage
\bigskip

\renewcommand{\thefigure}{\theenumi}
\renewcommand{\thetable}{\theenumi}
\section*{Question}
A charged 30 $\mu$F capacitor is connected to a 27 mH inductor. Suppose the initial charge on the capacitor is 6mC.What is the total energy stored in the circuit initially? What is the
total energy at later time?
\fi
\begin{figure}[h]


  
    

    \begin{circuitikz}[american]
    \draw (0,0)
    to[L, l=$L$ , i=$i\brak{t}$] (2,0)
    to[C, l=$C$, v=$V\brak{0}$] (2,-2)
    -- (0,-2)
    --(0,0);
\end{circuitikz}




  
  
       
  
    \caption{Circuit diagram}
     
   \label{fig:12.7.8.1}
\end{figure}

\begin{table}[h]
  \centering
  \begin{tabular}{|c|c|c|}
    \hline
    Symbol & Description & Value\\
    \hline
    q\brak{0^{+}} & Initial charge on capacitor & 6\ mC \\
    \hline
    q\brak{t}&Charge on capacitor&-\\
    \hline
    L & Value of inductance & 27\ mH \\
    \hline
    C & Value of capacitance & $30\ \mu F$ \\
    \hline
    E&Total energy stored in circuit&-\\
    \hline
    $E_L$&Energy stored in inductor&-\\
    \hline
    $E_C$&Energy stored in capacitor&-\\
    \hline
	i\brak{t}&current in the inductor&$\frac{dq}{dt}$\\
    \hline
    I\brak{s}&Laplace transform of i\brak{t}&-\\
    \hline
  \end{tabular}

  \caption{Table of parameters}
  \label{tab:12.7.8.1}
\end{table}
 \begin{figure}[h]


  
    

    \begin{circuitikz}[american]
    \draw (0,0)
    to[american inductor, l=$sL$, i=$i\brak{s}$] (2,0)
    to[capacitor, l=$\frac{1}{sC}$] (2,-2)
    -- (0,-2)
    to[V, v=$\frac{v(0^-)}{s}$] (0,0); 
\end{circuitikz}




  
  
       
  
    \caption{Circuit diagram in laplace domain}
     
   \label{fig:12.7.8.2}
\end{figure}
Writing KVL in above circuit,
\begin{align}
    L{sI\brak{s}}+\frac{v\brak{0^{+}}}{s}+\frac{1}{C}\frac{I\brak{s}}{s}=0
\end{align}
\begin{align}
    I\brak{s}= \frac{-v\brak{0^{+}}C}{LCs^2+1}
\end{align}
From initial value theorem ,
\begin{align}
    i\brak{0^{+}}=\lim_{s\to\infty}[sI\brak{s}]
\end{align}
\begin{align}
    i\brak{0^{+}}=\lim_{s\to\infty}\left[s\frac{-v\brak{0^{+}}C}{LCs^2+1}\right]=0
\end{align}
From final value theorem ,
\begin{align}
     i\brak{\infty}=\lim_{s\to0}[sI\brak{s}]
\end{align}
\begin{align}
   i\brak{\infty}=\lim_{s\to0}\left[s\frac{-v\brak{0^{+}}C}{LCs^2+1}\right]=0 
\end{align}
\begin{align}
	\label{12.7.8}
	i\brak{0^{+}}&=i\brak{\infty}\\
	&=0
\end{align}
Hence,
\begin{align}
	q\brak{0^{+}}&=q\brak{\infty}\\
	&=6\ mC
\end{align}

from \brak{\ref{12.7.8}},
\begin{align}
    E_L=0
\end{align}
\begin{gather}
\begin{align}
	E_C&=\frac{q^2}{2C}\\
       &=0.6\ J
\end{align}
\begin{align}
    E=E_L+E_C
\end{align}
\end{gather}
\begin{align}
    E=0.6\ J
\end{align}
Hence , the total energy stored in the circuit initially and at a later time is 0.6\ J.

